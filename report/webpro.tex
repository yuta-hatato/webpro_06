\documentclass[uplatex,dvipdfmx]{jsarticle}

% --- 日本語フォント/欧文・数式 ---
\usepackage[uplatex,deluxe]{otf} % 日本語フォント拡張(uplatex)
\usepackage[noalphabet]{pxchfon} % otfの後
\usepackage{stix2}               % 欧文&数式フォント
\usepackage[fleqn,tbtags]{mathtools}

% --- 図表・配置 ---
\usepackage{graphicx}
\usepackage{here}
\usepackage{tabularx}

% --- ソースコード ---
\usepackage{listings}

% --- PDF/リンク ---
\usepackage[dvipdfmx,
  bookmarks=true,
  bookmarksnumbered=true,
  colorlinks=true,
  linkcolor=blue,
  urlcolor=blue,
  citecolor=blue
]{hyperref}
\usepackage{pxjahyper} % hyperrefの後(日本語しおり対策)

% listings(ソースコード表示)の設定
\lstset{
  basicstyle={\ttfamily\small},
  identifierstyle={\small},
  commentstyle={\small\itshape},
  keywordstyle={\small\bfseries},
  ndkeywordstyle={\small},
  stringstyle={\small\ttfamily},
  frame={tb},
  breaklines=true,
  columns=[l]{fullflexible},
  numbers=left,
  xrightmargin=0zw,
  xleftmargin=3zw,
  numberstyle={\scriptsize},
  stepnumber=1,
  numbersep=1zw,
  lineskip=-0.5ex
}

% --- タイトル情報 ---
\title{一覧表示をベースとしたWebアプリケーションの作成(3テーマ)\\仕様書(利用者向け/管理者向け/技術者向け)}
\author{25G1107 畑戸悠太}
\date{\today}

\begin{document}
\maketitle
\tableofcontents
\newpage

%========================================================
\section{リポジトリ}
本課題のソースコードは GitHub 上のリポジトリで管理する.提出時点では未登録のため,後日 URL を追記する.

\begin{itemize}
  \item リポジトリURL:\url{https://github.com/XXXX/XXXX}(後日追記)
\end{itemize}

\subsection*{動作環境(共通)}
本アプリケーションは Node.js(Express)を用いた簡易APIサーバと,静的HTML(JavaScript)で構成される.
\begin{itemize}
  \item Node.js:18以上(目安)
  \item ブラウザ:Chrome / Edge / Safari 等のモダンブラウザ
  \item 起動:\texttt{node server.js}(\texttt{http://localhost:3000})
\end{itemize}

%========================================================
\section{千葉県市町村一覧(利用者向け仕様書)}
\subsection{概要}
千葉県内の市町村(市・町・村)を一覧表示し,クリックで詳細ページを閲覧できるWebアプリケーションである.
検索や区分(市・町・村)による絞り込みにより,目的の自治体を素早く探せる.

\subsection{利用方法}
\begin{enumerate}
  \item ブラウザで \texttt{/chiba/index.html} を開く.
  \item 画面上部の検索欄にキーワード(例:``佐倉'',``町'')を入力する.
  \item 区分(市・町・村)で絞り込みたい場合はプルダウンを選択する(実装している場合).
  \item 一覧の自治体名をクリックすると \texttt{/chiba/detail.html?id=...} に遷移し,詳細を確認できる.
\end{enumerate}

\subsection{画面構成}
\begin{itemize}
  \item 一覧画面(index):自治体名/区分/概要を表示し,詳細へ遷移できる.
  \item 詳細画面(detail):概要・詳細本文・参考リンク・画像(任意)を表示する.
\end{itemize}

\subsection{特徴(工夫点:山武市の強調表示)}
本アプリでは,特定の自治体(例:山武市)を目立たせるために,一覧画面でもカード装飾やラベル表示などにより強調している.
詳細画面では画像表示や長文説明により,情報量の差を視覚的に伝える.

% --- ここに「千葉 一覧画面」のスクリーンショットを入れる(コメントアウト) ---
% \begin{figure}[H]
%   \centering
%   \includegraphics[width=0.95\linewidth]{images/chiba_index.png}
%   \caption{千葉県市町村一覧(index画面)}
% \end{figure}

% --- ここに「山武市 詳細画面」のスクリーンショットを入れる(コメントアウト) ---
% \begin{figure}[H]
%   \centering
%   \includegraphics[width=0.95\linewidth]{images/chiba_sammu_detail.png}
%   \caption{山武市の詳細表示(detail画面)}
% \end{figure}

%========================================================
\section{千葉県市町村一覧(管理者向け仕様書)}
\subsection{管理対象}
管理者は,表示する自治体データ(JSON)および画像ファイルを管理する.
\begin{itemize}
  \item データ:\texttt{chiba-municipalities/items.json}
  \item 画面:\texttt{public/chiba/index.html}, \texttt{public/chiba/detail.html}
  \item 画像:\texttt{public/chiba/images/}(推奨)
\end{itemize}

\subsection{データ更新手順}
\begin{enumerate}
  \item \texttt{chiba-municipalities/items.json} を編集する.
  \item 各項目は \texttt{id, name, category, summary, detail, link} を持つ(必要に応じて追加項目も可).
  \item サーバを再起動する(またはリロードで反映する).
\end{enumerate}

\subsection{画像の追加}
自治体ごとに画像を表示する場合は,\texttt{public/chiba/images/} に画像を置き,JSONの \texttt{image} 等のキーで参照する方式を推奨する.
(例:\texttt{image: "/chiba/images/sammu.png"})

% --- ここに「管理のための items.json 編集例」の図を入れる(コメントアウト) ---
% \begin{figure}[H]
%   \centering
%   \includegraphics[width=0.95\linewidth]{images/chiba_items_json.png}
%   \caption{items.json の編集例}
% \end{figure}

\subsection{起動・停止}
\begin{itemize}
  \item 起動:\texttt{node server.js}
  \item 停止:ターミナルで \texttt{Ctrl + C}
\end{itemize}

\subsection{トラブルシュート}
\begin{itemize}
  \item 画面が真っ白:\texttt{server.js} が起動しているか確認する.
  \item 一覧が出ない:\texttt{/api/chiba/items} がJSONを返すか確認する.
  \item 詳細が出ない:\texttt{id} がJSON内に存在するか確認する.
\end{itemize}

%========================================================
\section{千葉県市町村一覧(技術者向け仕様書)}
\subsection{システム構成}
\begin{itemize}
  \item サーバ:Node.js + Express
  \item API:\texttt{/api/chiba/items}, \texttt{/api/chiba/items/:id}
  \item フロント:\texttt{public/chiba/index.html}, \texttt{public/chiba/detail.html}
  \item データ:\texttt{chiba-municipalities/items.json}
\end{itemize}

\subsection{ディレクトリ設計意図}
\begin{itemize}
  \item \texttt{public/} は静的配信対象(ブラウザから直接参照される).
  \item \texttt{*-municipalities/} はデータ置き場(サーバが読み込む).
  \item ``一覧表示ベース'' の要件に対し,\textbf{一覧(index)→詳細(detail)}の2画面構成を統一した.
\end{itemize}

\subsection{API仕様}
\paragraph{一覧取得}
\texttt{GET /api/chiba/items}
\begin{itemize}
  \item レスポンス:items.json の配列
\end{itemize}

\paragraph{詳細取得}
\texttt{GET /api/chiba/items/:id}
\begin{itemize}
  \item パス:\texttt{id} は数値
  \item レスポンス:一致する1件のJSON
\end{itemize}

\subsection{フロント実装の要点}
\begin{itemize}
  \item \texttt{fetch()} でAPIからJSONを取得し,DOMに反映する.
  \item 一覧の各要素は \texttt{detail.html?id=...} の形式で遷移する.
  \item 特定条件(例:\texttt{id==35} の山武市)でクラス付与し,CSSで強調できる.
\end{itemize}

\subsection{server.js(抜粋)}
% --- 必要に応じて、実際の server.js を貼る(コメントアウト例) ---
% \begin{lstlisting}[language=JavaScript,caption={server.js(抜粋)}]
% // ここに server.js の該当部分
% \end{lstlisting}

%========================================================
\section{千葉工大施設一覧(利用者向け仕様書)}
\subsection{概要}
千葉工業大学(対象キャンパス)の施設情報を一覧表示し,クリックで詳細を閲覧できるWebアプリケーションである.
学内施設の名称と概要を簡潔に確認できることを目的とする.

\subsection{利用方法}
\begin{enumerate}
  \item ブラウザで \texttt{/cit/index.html} を開く.
  \item 施設名をクリックして詳細画面へ遷移する.
  \item 必要に応じて検索欄から絞り込む(実装している場合).
\end{enumerate}

\subsection{画面構成}
\begin{itemize}
  \item 一覧画面(index)
  \item 詳細画面(detail)
\end{itemize}

% --- ここに「CIT 一覧画面」のスクリーンショットを入れる(コメントアウト) ---
% \begin{figure}[H]
%   \centering
%   \includegraphics[width=0.95\linewidth]{images/cit_index.png}
%   \caption{千葉工大施設一覧(index画面)}
% \end{figure}

%========================================================
\section{NPB球団一覧(利用者向け仕様書)}
\subsection{概要}
NPB(12球団)を一覧表示し,クリックで詳細ページを閲覧できるWebアプリケーションである.
球団名・所属リーグ・概要に加え,\textbf{最後に日本一になった年}などの情報を表示する.

\subsection{利用方法}
\begin{enumerate}
  \item ブラウザで \texttt{/npb/index.html} を開く.
  \item 検索欄で球団名やリーグ名を入力し,絞り込む.
  \item 球団名をクリックすると詳細画面へ遷移する.
\end{enumerate}

\subsection{画面構成}
\begin{itemize}
  \item 一覧画面(index):検索・リーグ絞り込み・日本一年度表示
  \item 詳細画面(detail):概要・詳細・参考リンク・日本一年度(強調表示)
\end{itemize}

% --- ここに「NPB 一覧画面」のスクリーンショットを入れる(コメントアウト) ---
% \begin{figure}[H]
%   \centering
%   \includegraphics[width=0.95\linewidth]{images/npb_index.png}
%   \caption{NPB球団一覧(index画面)}
% \end{figure}

% --- ここに「NPB 詳細画面」のスクリーンショットを入れる(コメントアウト) ---
% \begin{figure}[H]
%   \centering
%   \includegraphics[width=0.95\linewidth]{images/npb_detail.png}
%   \caption{NPB球団詳細(detail画面)}
% \end{figure}

\subsection{備考(データの出典について)}
最後に日本一になった年は外部資料の結果表等を参考に入力した.
(提出時は,参照元URL等を追記する.)

%========================================================
\section*{付録:提出物}
\begin{itemize}
  \item PDF:\texttt{wpro2025.pdf}
  \item 内容:本書(仕様書:千葉は3部,CIT/NPBは利用者向けのみ)
\end{itemize}

\end{document}
